\documentclass[12pt]{article}

\usepackage[parfill]{parskip}
\usepackage[tight]{subfigure}
\usepackage{amsmath}
\usepackage{amssymb}
\usepackage{amsthm}
\usepackage{fullpage}
\usepackage{hyperref}
\usepackage{latexsym}
\usepackage{mathtools}

\DeclareMathOperator*{\tr}{\mathrm{tr}}
\DeclareMathOperator*{\diag}{\mathrm{diag}}

\title{Sparse linear-quadratic regulator}
\author{Matt Wytock}

\begin{document}

\maketitle

\section{Problem}
We have a system defined by
\begin{equation}
\begin{split}
\dot{x} &= Ax + Bu + w
\end{split}
\end{equation}
where $w$ is random with covariance $E[ww^T] = W$. We wish to find a feedback
controller
\begin{equation}
u = Kx
\end{equation}
minimizing the quadratic objective
\begin{equation}
J = \int_0^\infty \left[ x(\tau)^TQx(\tau) + u(\tau)^TRu(\tau) \right] d\tau.
\end{equation}
This is equivalent to
\begin{equation}
J = \tr WP(K)
\end{equation}
where $K \in \mathcal{D}_s$, the set of stabilizing feedback gains
\begin{equation}
\mathcal{D}_s = \{ K | Re\{\lambda(A + BK)\} < 0 \}
\end{equation}
and $P(K)$ is the unique solution to the Lyapunov equation
\begin{equation}
(A + BK)^TP + P(A + BK) + (Q + K^TRK) = 0.
\end{equation}

\section{Differentials}
We begin with the differential for the matrix-valued function $P(K)$. Let
\begin{equation}
\psi(P,K) = (A + BK)^TP + P(A + BK) + (Q + K^TRK)
\end{equation}
and we have the partials
\begin{equation}
\begin{split}
\psi_P(P,K) &= (A + BK)^TdP + dP(A + BK) \\
\psi_K(P,K) &= (BdK)^TP + PBdK + dK^TRK + K^TRdK
\end{split}
\end{equation}
which for a given $K$ and $dK$ results in the Lyapunov equation
\begin{equation}
\label{eq-lyap-pdiff}
(A + BK)^T\tilde{P} + \tilde{P}(A + BK) + dK^T(B^TP + RK) + (PB + K^TR)dK = 0
\end{equation}
where $P = P(K)$ and $\tilde{P} = P'(K)dK$ is the unique solution. This gives us an
expression for the differential of the objective
\begin{equation}
J'(K)dK = \tr P'(K)dKW.
\end{equation}
However, in order to write the derivative in gradient form we need an
expression
\begin{equation}
J'(K)dK = \tr \nabla J(K)^TdK.
\end{equation}
Let $L$ be the solution to the Lyapunov equation
\begin{equation}
\label{eq-lyap-l}
(A + BK)L + L(A + BK)^T + W = 0
\end{equation}
and by multiplying \eqref{eq-lyap-pdiff} by $L$, \eqref{eq-lyap-l} by $\tilde{P}$ and taking
the trace we get
\begin{equation}
\label{eq-diff1}
\tr P'(K) dKW = 2 \tr L(PB + K^TR)dK
\end{equation}
which results in our equation for the derivative
\begin{equation}
\nabla J(K) = 2(B^TP + RK)L
\end{equation}
Continuing, we can compute the second differential from \eqref{eq-diff1} as
\begin{equation}
\begin{split}
J''(K)(dK,dK) &= 2 \tr  [ L'(K)dK(P(K)B + K^TR)dK  + \\
&  \quad L(K)(P'(K)dKB + dK^TR)dK ] \\
&= 2 \tr H(K,dK) dK
\end{split}
\end{equation}
where
\begin{equation}
H(K,dK) = \tilde{L}(PB + K^TR) + L(\tilde{P}B + dK^TR)
\end{equation}
and $\tilde L = L'(K)dK$ and $\tilde P = P'(K) dK$ are the unique solutions to
\eqref{eq-lyap-pdiff} and
\begin{equation}
\label{eq-lyap-ldiff}
(A + BK)\tilde{L} + \tilde{L}(A + BK)^T + BdKL + LdK^TB^T = 0
\end{equation}
respectively.

\section{Second-order approximation}

In this section we derive an explicit form of the second-order approximation
which we will need for the coordinate descent updates and to write down the
Hessian. Let $D$ be an arbitrary direction, the second order approximation at $K$ is
given by
\begin{equation}
\label{eq-approx}
g(K,D) = 2 \tr L(PB + K^TR)D + \tr \tilde{L}(PB + K^TR)D + \tr L(\tilde{P}B + D^TR)D.
\end{equation}
where $\tilde{L}$ and $\tilde{P}$ are the solutions to Lyapunov equations that
depend on $D$. In general notation they have the form
\begin{equation}
AX + XA^T + Q(D) = 0
\end{equation}
which we can solve for $X$ by taking the eigendecomposition $A = U\Lambda U^{-1}$. This gives
\begin{equation}
\begin{split}
U\Lambda U^{-1}X + XU^{-T}\Lambda U^T + Q(D) &= 0 \\
\Lambda U^{-1}XU^{-T}X + U^{-1}XU^{-T}\Lambda &= -U^{-1}Q(D)U^{-T}
\end{split}
\end{equation}
which we can solve by noting that $\Lambda$ is diagonal with elements
$\lambda_1,\ldots,\lambda_n$ and thus
\begin{equation}
\label{eq-lyap-explicit}
X = U(-U^{-1}Q(D)U^{-T} \circ \Theta)U^T
\end{equation}
where $\Theta_{ij} = 1 /(\lambda_i + \lambda_j)$. Applying the above to the
Lyapunov equations \eqref{eq-lyap-pdiff} and  \eqref{eq-lyap-ldiff} we get the
explicit forms
\begin{equation}
\label{eq-lyap-explicit-diff}
\begin{split}
\tilde{P} &= -U^{-T} \left(U^T (D^T(B^TP + RK) + (PB +
K^TR)D) U \circ \Theta \right)U^{-1}, \\
\tilde{L} &= -U \left( U^{-1} (BDL + LD^TB^T) U^{-T} \circ \Theta \right) U^T
\end{split}
\end{equation}
where $U\Lambda U^{-1}$ is the eigendecomposition of $A + BK$.

Since $\Theta$ is complex symmetric we can take the Autonne-Takagi factorization and
write $\Theta = U\Sigma U^T = XX^T$. Furthermore, we typically have that
$\Theta$ is low rank and thus we can take $XX^T \approx \sum_{i=1}^k x_ix_i^*$
with $k \ll n$.

\section{Coordinate descent updates}

Let $E = PB + K^TR$ and we can rewrite \eqref{eq-approx} as
\begin{equation}
g(K,D) = 2 \tr LED + \tr \tilde{L}ED + \tr L\tilde{P}BD + \tr LD^TRD.
\end{equation}
Plugging in the the explicit solutions to the Lyapunov equations from
\eqref{eq-lyap-explicit-diff}, we have
\begin{equation}
\begin{split}
\tr \tilde{L}ED &= - \tr U \left( U^{-1} (BDL + LD^TB^T) U^{-T} \circ \Theta \right) U^TED \\
&\approx - \tr  U\left( \sum_{i=1}^k \diag(x_i) U^{-1}(BDL + LD^TB^T)U^{-T})
  \diag(x_i) \right) U^TED \\
&= \sum_{i=1}^k -\tr X_i BDL X_i^TED  - \tr X_i(BDL)^TX_i^TED
\end{split}
\end{equation}
where $X_i = U\diag(x_i)U^{-1}$. We also have
\begin{equation}
\begin{split}
\tr L\tilde{P}BD &= -\tr U^{-T} \left(U^T (D^TE^T + ED) U \circ \Theta \right)U^{-1}BDL \\
&\approx -\tr U^{-T} \left(\sum_{i=1}^k \diag(x_i) U^T (D^TE^T + ED) U \diag(x_i) \right)U^{-1}BDL \\
&= \sum_{i=1}^k -\tr X_i^T(ED)^TX_iBDL - \tr X_i^TED X_i BDL
\end{split}
\end{equation}
Noting that these are equivalent, we can rewrite the second-order approximation as
\begin{equation}
g(K,D) = 2 \tr LED + \tr LD^TRD - 2 \left( \sum_{i=1}^k \tr X_i^T(ED)^TX_iBDL + \tr X_i^TED X_i BDL \right).
\end{equation}
To derive the coordinatewise updates let $D = D + \mu e_ie_j^T$ and we
take the minimum over $\mu$. First, we consider the terms inside the sum and
drop the subscript on $X_i$ for brevity and for the first term we have
\begin{equation}
\begin{split}
\arg \min_\mu & \tr  X^T(D + \mu e_ie_j^T)^TE^TXB(D + \mu e_ie_j^T)L \\
= \arg \min_\mu & \mu \tr X^Te_je_i^T E^TXBDL + \mu \tr X^TD^TE^TXB e_ie_j^TL +
\mu^2 X^Te_je_i^TE^TXBe_ie_j^TL \\
= \arg\min_\mu & \mu (E^TXBDLX^T)_{ij} + \mu(B^TX^TEDXL)_{ij} + \mu^2(E^TXB)_{ii}(LX^T)_{jj}.
\end{split}
\end{equation}
Similarly, for the second term we have
\begin{equation}
\begin{split}
\arg\min_\mu  &\tr X^TE(D +\mu e_ie_j^T) X B(D + \mu e_ie_j^T)L \\
= \arg\min_\mu & \mu \tr X^T E e_ie_j^TXBDL + \mu \tr X^TEDXBe_ie_j^TL + \mu^2
X^TEe_ie_j^TXBe_ie_j^TL \\
= \arg\min_\mu & \mu (XBDLX^TE)_{ji} + \mu (LX^TEDXB)_{ji} + \mu^2 (LX^TE)_{ji}(XB)_{ji}.
\end{split}
\end{equation}
Finally, for the first terms outside of the sum we have
\begin{equation}
\begin{split}
\arg\min_\mu  & 2 \tr LE(D + \mu e_ie_j^T) + \tr L(D + \mu e_ie_j^T)^TR(D + \mu e_ie_j^T)  \\
= \arg\min_\mu & 2 \mu \tr LEe_ie_j^T + 2 \mu \tr Le_je_i^TRD + \mu^2 \tr
Le_je_i^TRe_ie_j^T \\
= \arg\min_\mu & 2 \mu (LE)_{ji} + 2\mu (RDL)_{ij} + \mu^2 L_{jj}R_{ii}
\end{split}
\end{equation}
Combining these expressions and adding $\ell_1$ regularization, we have
\begin{equation}
\arg \min_\mu g(K, D + \mu e_ie_j^T) + \lambda\|K + D + \mu e_ie_j^T\|_1 = \arg
\min_\mu \frac{1}{2} a^2 \mu + b \mu + \lambda \|c + \mu \|_1
\end{equation}
where
\begin{equation}
\begin{split}
a &= 2 R_{ii}L_{jj} - 4 \left( \sum_{k=1}^r (E^TX_kB)_{ii}(LX_k^T)_{jj}
+ (LX^TE)_{ji}(XB)_{ji} \right) \\
b &= 2 (E^TL)_{ij} + 2(RDL)_{ij} \\
& - 2\left( \sum_{k=1}^r (E^TX_kBDLX_k^T)_{ij} +
(B^TX_k^TEDX_kL)_{ij} + (X_kBDLX_k^TE)_{ji} + (LX_k^TEDX_kB)_{ji} \right) \\
c &= K_{ij} + D_{ij}
\end{split}
\end{equation}
which has the closed form solution
\begin{equation}
\mu = -c + S_{\lambda/a}\left( c - \frac{b}{a} \right)
\end{equation}
where $S_\lambda$ is the soft-thresholding operator.


\end{document}
